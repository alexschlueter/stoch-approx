\documentclass[ngerman,a4paper,11pt]{scrartcl}
\usepackage{lmodern}
\usepackage[utf8]{inputenc}
\usepackage{babel}
\usepackage{csquotes}
\usepackage{bbm}
\usepackage[T1]{fontenc}
\usepackage{braket}
\usepackage{enumitem}
\usepackage[shortcuts]{extdash}
\usepackage{cancel}
\usepackage[backend=biber,style=alphabetic]{biblatex}
\usepackage{nameref}
\usepackage{hyperref}
\usepackage{amsmath,amsthm,amssymb}
\usepackage{mathtools}
\usepackage{thmtools}
\usepackage[capitalize]{cleveref}

\bibliography{literatur.bib}

\newlist{thmlist}{enumerate}{1}
\setlist[thmlist]{label=(\roman{thmlisti}), ref=\thethm (\roman{thmlisti}),noitemsep}
\newlist{remlist}{enumerate}{1}
\setlist[remlist]{label=(\arabic*), ref=\thethm (\arabic*),noitemsep}
\newlist{thmasslist}{enumerate}{1}
\setlist[thmasslist]{label=(\alph*), ref=\thethm (\alph*),noitemsep}

\newtheoremstyle{break}{}{}{}{}{\bfseries}{}{\newline}{}
\declaretheoremstyle[headpunct=.:]{claim}

\declaretheorem[
name=Satz,
%refname={theorem,theorems},        %Lower Case Versions of Theorem Type
% Refname={Satz,Sätze},
style=definition,
numberwithin=section]{thm}

\declaretheorem[
name=Satz,
%refname={theorem,theorems},        %Lower Case Versions of Theorem Type
% Refname={Satz,Sätze},
style=break,
sibling=thm]{thmbk}

\declaretheorem[
name=Lemma,
%refname={lemma,lemmas},
% Refname={Lemma,Lemmas},
style=definition,
sibling=thm]{lem}

\declaretheorem[
name=Definition,
%refname={lemma,lemmas},
% Refname={Definition,Definitionen},
style=definition,
sibling=thm]{defn}

\declaretheorem[
name=Bemerkung,
%refname={lemma,lemmas},
% Refname={Bemerkung,Bemerkungen},
style=definition,
sibling=thm]{rem}

\declaretheorem[
name=Beispiel,
%refname={lemma,lemmas},
% Refname={Bemerkung,Bemerkungen},
style=definition,
sibling=thm]{exmp}

\declaretheorem[
name=Beispiel,
%refname={theorem,theorems},        %Lower Case Versions of Theorem Type
% Refname={Satz,Sätze},
style=break,
sibling=thm]{exmpbk}

\declaretheorem[
name=Beh,
numbered=no,
%refname={lemma,lemmas},
% Refname={Bemerkung,Bemerkungen},
style=claim]{claim}

\declaretheorem[
name=Beweis,
numbered=no,
%refname={lemma,lemmas},
% Refname={Bemerkung,Bemerkungen},
qed=\ensuremath{\Diamond},
style=remark]{dproof}

% \Crefname{thm}{Satz}{Sätze}
% \Crefname{lem}{Lemma}{Lemmas}

\addtotheorempostheadhook[thm]{\crefalias{thmlisti}{thm}}
\addtotheorempostheadhook[lem]{\crefalias{thmlisti}{lem}}
\addtotheorempostheadhook[rem]{\crefalias{remlisti}{rem}}

% \theoremstyle{definition}
% \newtheorem{thm}{Satz}[section]
% \newtheorem{lem}[thm]{Lemma}
% \newtheorem{prop}[thm]{Proposition}
% \newtheorem{cor}[thm]{Korollar}

% \newtheorem{defn}[thm]{Definition}
% \newtheorem{conj}[thm]{Vermutung}
% \newtheorem{exmp}[thm]{Beispiel}

% % \theoremstyle{remark}
% \newtheorem{rem}[thm]{Bemerkung}
% \newtheorem*{note}{Merke}
\DeclarePairedDelimiterX{\norm}[1]{\lVert}{\rVert}{#1}
\newcommand{\fracnorm}[3]{%
    \ensuremath{\frac{\! #1\!}{\! {}_{\phantom{#3}}\norm{#2}_{#3}\!}}%
}
\newcommand\coloniff{\vcentcolon\!\iff}

\makeatletter
\newcommand*{\transpose}{%
  {\mathpalette\@transpose{}}%
}
\newcommand*{\@transpose}[2]{%
  % #1: math style
  % #2: unused
  \raisebox{\depth}{$\m@th#1\intercal$}%
}
\makeatother

\newcommand{\stcomp}[1]{{#1}^{\mathsf{c}}} % Mengenkomplement
\newcommand{\CC}{\mathbb{C}}
\newcommand{\EE}{\mathbb{E}}
\newcommand{\FF}{\mathbb{F}}
\newcommand{\HH}{\mathbb{H}}
\newcommand{\KK}{\mathbb{K}}
\newcommand{\NN}{\mathbb{N}}
\newcommand{\OO}{\mathbb{O}}
\newcommand{\QQ}{\mathbb{Q}}
\newcommand{\RR}{\mathbb{R}}
\newcommand{\ZZ}{\mathbb{Z}}
\newcommand{\bb}{\mathcal{B}}
\newcommand{\cc}{\mathcal{C}}
\newcommand{\ff}{\mathcal{F}}
\renewcommand{\ll}{\mathcal{L}}
\newcommand{\nn}{\mathcal{N}}
\newcommand{\zz}{\mathcal{Z}}

\newcommand{\Cb}[1]{C_b(#1)}
\newcommand{\expect}[1]{\EE[#1]}
\newcommand{\dvar}[1]{\,\mathrm{d}#1}
\newcommand{\dmeas}[2]{\,#1(\mathrm{d}#2)}
% \newcommand{\meas}[2]{\,\mathrm{d}#1(#2)}
\DeclarePairedDelimiter{\sprod}{\langle}{\rangle}	% spitze Klammern
\DeclarePairedDelimiter{\abs}{\lvert}{\rvert}		% Betrag
\DeclareMathOperator{\ggT}{ggT}

\newcommand\phantomarrow[2]{%
  \setbox0=\hbox{$\displaystyle #1\to$}%
  \hbox to \wd0{%
    $#2\mapstochar
     \cleaders\hbox{$\mkern-1mu\relbar\mkern-3mu$}\hfill
     \mkern-7mu\rightarrow$}%
  \,}

\begin{document}

\title{Konvergenzraten des Robbins-Monro-Algorithmus}
\author{Alexander Schlüter}
% \thanks{Dozent: Prof.~Dr.~Dereich, Betreuung: Johannes Blank}
\date{13. Juli 2016}
% Bachelorseminar Markovketten, WS 15/16
% {\let\newpage\relax\maketitle}
\maketitle
\begin{abstract}
  Der Robbins-Monro-Algorithmus ist ein iteratives Verfahren zur Bestimmung von
  Nullstellen einer Funktion, deren Wert nur stochastisch gestört gemessen
  werden kann. Im letzten Vortrag wurde die fast sichere Konvergenz gegen eine
  Nullstelle bewiesen. Ziel dieses Vortrages ist es, eine Aussage über die
  Konvergenzgeschwindigkeit und die Form der Verteilung im Limit ähnlich dem
  zentralen Grenzwertsatz zu beweisen.
\end{abstract}
% \newpage
\tableofcontents
% \newpage

\section{Erinnerungen}
Im Folgenden sei    
\begin{itemize}
\item $(\Omega, \ff, P)$ ein Wahrscheinlichkeitsraum und $(\zz,\cc)$ ein messbarer Raum,
\item $(Z_n)_{n\geq 1}$ eine Folge u.i.v $(\zz, \cc)$-wertiger Zufallsvariablen,
\item $I\subset\RR$ ein abgeschlossenes Intervall oder $I=\RR$, 
\item $X_0$ eine von $(Z_n)_{n\geq 1}$ unabhängige $I$-wertige Zufallsvariable,
\item $(\gamma_n)_{n\geq 1}$ eine Folge in $(0,\infty)$ und
\item $H:(I\times \zz,\bb(I)\otimes\cc)\to(\RR,\bb(\RR))$ eine messbare
  Abbildung mit
  \begin{equation*}
    H(x,Z_1)\in\ll^1\quad\text{für alle $x\in I$.}
  \end{equation*}
\end{itemize}

Der \textbf{Robbins-Monro-Algorithmus} ist die Folge $X=(X_n)_{n\geq 0}$ von
reellen Zufallsvariablen definiert durch die Rekursion
\begin{equation}
  \label{eq:algo}
  X_{n+1}=X_n+\gamma_{n+1}H(X_n,Z_{n+1})\,,
\end{equation}
wobei wir annehmen, dass die Werte immer in $I$ liegen.

Die \textbf{Erwartungswertfunktion} des Algorithmus ist definiert durch
\begin{equation*}
 h:I\to\RR,\quad h(x)\coloneqq\expect{H(x,Z_1)}
\end{equation*}
und außerdem sei
\begin{equation*}
 g:I\to\overline{\RR}_+,\quad g(x)\coloneqq\expect{H(x,Z_1)^2}.
\end{equation*}

Wir haben gesehen, dass der Robbins-Monro-Algorithmus unter zusätzlichen
Annahmen fast sicher gegen eine Nullstelle von $h$ konvergiert. Diese Annahmen
waren
\begin{enumerate}[label=(\roman*)]
\item (Downcrossing-Bedingung) $h$ ist stetig und es gibt ein $x_0\in h^{-1}(0)$ mit
  \begin{equation*}
   \sup_{x\in I}(x-x_0)h(x)\leq 0\,,
  \end{equation*}
\item (Beschränkte Störung) $X_0\in\ll^2$ und $g(x)\leq C(1+x^2)$ für eine Konstante $C\in\RR_+$ und
  alle $x\in I$,
\item (Abnehmende Schrittweiten) $\sum_{n=1}^\infty \gamma_n=\infty$ und $\sum_{n=1}^\infty\gamma_n^2<\infty$.
\end{enumerate}

\section{Vorbereitungen}
\begin{defn}
  % Es sei $K$ ein Markov-Kern und $(K_n)_{n\geq 1}$ eine Folge von Markov-Kernen von
  % $(\Omega, \ff)$ nach $(\RR^d, \bb(\RR^d))$. Außerdem sei $(X_n)_{n\geq 1}$
  % eine Folge von Zufallsvariablen von $(\Omega, \ff)$ nach $(\RR^d, \bb(\RR^d))$.
  % \begin{thmlist}
  % \item Die Folge $(K_n)_{n\geq 1}$ heißt \textbf{schwach konvergent} gegen $K$ und wir
  %   schreiben
  %   \begin{equation*}
  %     K_n\overset{w}{\longrightarrow} K\, ,
  %   \end{equation*}
  %   falls für alle $f\in\Li{P}$ und $h\in\Cb{\RR^d}$ gilt
  %   \begin{equation*}
  %    \lim_{n\to\infty}\int f(\omega)h(x)\, P\otimes K_n(\mathrm{d}(\omega,x)) = \int f(\omega)h(x)\, P\otimes K(\mathrm{d}(\omega,x))\, .
  %   \end{equation*}
  %   \item Die Folge $(X_n)_{n\geq 1}$ heißt \textbf{stabil konvergent} gegen $K$
  %     und wir schreiben
  %     \begin{equation*}
  %       X_n\longrightarrow K
  %     \end{equation*}
  % \end{thmlist}
  Es sei $(Y_n)_{n\geq 1}$ eine Folge von Zufallsvariablen von $(\Omega, \ff)$
  nach $(\RR^d, \bb(\RR^d))$ und $\nu$ ein Wahrscheinlichkeitsmaß auf
  $\bb(\RR^d)$. $(Y_n)_{n\geq 1}$ heißt \textbf{mischend konvergent} gegen $\nu$
  und wir schreiben
  \begin{equation*}
    Y_n\longrightarrow\nu \quad\textit{mischend}, 
  \end{equation*}
  falls für alle $f\in\ll^1(P)$ und $h\in\Cb{\RR^d}$ gilt
  \begin{equation*}
   \lim_{n\to\infty} \expect{f\cdot h(Y_n)} = \int f\dvar{P}\int h\dvar{\nu}\,.
  \end{equation*}
\end{defn}
\begin{rem}
 Durch Wahl von $f=1$ folgt aus der mischenden Konvergenz die bekannte
 Konvergenz in Verteilung. 
\end{rem}
\begin{thm}[Stabiler CLT]
 Seien $\gamma_n=C_1/(C_2+n)$ für $n\geq 1$ mit reellen Konstanten $C_1>0$,
 $C_2\geq 0$ und $x_0\in h^{-1}(0)$. Die folgenden Bedingungen seien erfüllt:
 \begin{thmasslist}
 \item $X_n\to x_0$ f.s. für $n\to\infty$.
 \item $X_0\in\ll^2$ und $g(x)\leq C(1+x^2)$ für eine Konstante $C\in\RR_+$ und
   alle $x\in I$,
 \item $g$ sei stetig in $x_0$, h sei differenzierbar in $x_0$, $h'(x_0)<0$ und
   \begin{equation*}
    h(x) = h'(x_0)(x-x_0) + \mathcal{O}((x-x_0)^2)\quad\text{für $x\to x_0$}\,,
   \end{equation*}
 \item Es existieren $\epsilon,\delta >0$ sodass $H(x,Z_1)\in\ll^{2+\delta}$ für
   alle $x\in I$ und 
   \begin{equation*}
    \sup_{\abs{x-x_0}\leq\epsilon}\expect{\abs{H(x,Z_1)}^{2+\delta}}<\infty\,.
   \end{equation*}
 \end{thmasslist}
 Dann gilt für $n\to\infty$:
 \begin{thmlist}
 \item Falls $\abs{h'(x_0)}C_1>1/2$:
   \begin{equation*}
    \sqrt{n}(X_n-x_0)\to\nn\left( 0,\frac{g(x_0)C_1^2}{2\abs{h'(x_0)}C_1-1} \right)\quad\textit{mischend,}
   \end{equation*}
 \item Falls $\abs{h'(x_0)}C_1=1/2$:
   \begin{equation*}
    \sqrt{\frac{n}{\log n}}(X_n-x_0)\to\nn\left( 0,g(x_0)C_1^2 \right)\quad\textit{mischend,}
   \end{equation*}
 \item Falls $\abs{h'(x_0)}C_1<1/2$: 
   \begin{equation*}
     n^{\abs{h'(x_o)}C_1}(X_n-x_0)\to\xi \quad\text{f.s.}
   \end{equation*}
   für eine reelle Zufallsvariable $\xi$, die von $X_0$ abhängt.
 \end{thmlist}
\end{thm}
\begin{rem}
  Downcrossing etc.
\end{rem}
\printbibliography

\end{document}