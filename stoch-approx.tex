\documentclass[ngerman,a4paper,11pt]{scrartcl}
\usepackage[T1]{fontenc}
\usepackage[utf8]{inputenc}
\usepackage{lmodern}
\usepackage{babel}
\usepackage{csquotes}
\usepackage{xfrac}
\usepackage{bbm}
\usepackage{braket}
\usepackage{enumitem}
\usepackage[shortcuts]{extdash}
\usepackage{cancel}
\usepackage[backend=biber,style=alphabetic]{biblatex}
\usepackage{nameref}
\usepackage{hyperref}
\usepackage{amsmath,amsthm,amssymb}
\usepackage{mathtools}
\usepackage{thmtools}
\usepackage[capitalize]{cleveref}

\bibliography{literatur.bib}

\newlist{thmlist}{enumerate}{1}
\setlist[thmlist]{label=(\roman{thmlisti}), ref=\thethm (\roman{thmlisti}),noitemsep}
\newlist{remlist}{enumerate}{1}
\setlist[remlist]{label=(\arabic*), ref=\thethm (\arabic*),noitemsep}
\newlist{thmasslist}{enumerate}{1}
\setlist[thmasslist]{label=(\alph*),ref=(\alph*),noitemsep}
\crefname{thmasslisti}{Voraussetzung}{Voraussetzungen}

\newtheoremstyle{break}{}{}{}{}{\bfseries}{}{\newline}{}
\declaretheoremstyle[headpunct=.:]{claim}

\declaretheorem[
name=Satz,
%refname={theorem,theorems},        %Lower Case Versions of Theorem Type
% Refname={Satz,Sätze},
style=definition,
numberwithin=section]{thm}

\declaretheorem[
name=Korollar,
%refname={theorem,theorems},        %Lower Case Versions of Theorem Type
% Refname={Satz,Sätze},
style=definition,
sibling=thm]{cor}

\declaretheorem[
name=Satz,
%refname={theorem,theorems},        %Lower Case Versions of Theorem Type
% Refname={Satz,Sätze},
style=break,
sibling=thm]{thmbk}

\declaretheorem[
name=Lemma,
%refname={lemma,lemmas},
% Refname={Lemma,Lemmas},
style=definition,
sibling=thm]{lem}

\declaretheorem[
name=Definition,
%refname={lemma,lemmas},
% Refname={Definition,Definitionen},
style=definition,
sibling=thm]{defn}

\declaretheorem[
name=Bemerkung,
%refname={lemma,lemmas},
% Refname={Bemerkung,Bemerkungen},
style=definition,
sibling=thm]{rem}

\declaretheorem[
name=Beispiel,
%refname={lemma,lemmas},
% Refname={Bemerkung,Bemerkungen},
style=definition,
sibling=thm]{exmp}

\declaretheorem[
name=Beispiel,
%refname={theorem,theorems},        %Lower Case Versions of Theorem Type
% Refname={Satz,Sätze},
style=break,
sibling=thm]{exmpbk}

\declaretheorem[
name=Beh,
numbered=no,
%refname={lemma,lemmas},
% Refname={Bemerkung,Bemerkungen},
style=claim]{claim}

\declaretheorem[
name=Beweis,
numbered=no,
%refname={lemma,lemmas},
% Refname={Bemerkung,Bemerkungen},
qed=\ensuremath{\Diamond},
style=remark]{dproof}

% \Crefname{thm}{Satz}{Sätze}
% \Crefname{lem}{Lemma}{Lemmas}

\addtotheorempostheadhook[thm]{\crefalias{thmlisti}{thm}}
\addtotheorempostheadhook[lem]{\crefalias{thmlisti}{lem}}
\addtotheorempostheadhook[rem]{\crefalias{remlisti}{rem}}

% \theoremstyle{definition}
% \newtheorem{thm}{Satz}[section]
% \newtheorem{lem}[thm]{Lemma}
% \newtheorem{prop}[thm]{Proposition}
% \newtheorem{cor}[thm]{Korollar}

% \newtheorem{defn}[thm]{Definition}
% \newtheorem{conj}[thm]{Vermutung}
% \newtheorem{exmp}[thm]{Beispiel}

% % \theoremstyle{remark}
% \newtheorem{rem}[thm]{Bemerkung}
% \newtheorem*{note}{Merke}
\DeclarePairedDelimiterX{\norm}[1]{\lVert}{\rVert}{#1}
\newcommand{\fracnorm}[3]{%
    \ensuremath{\frac{\! #1\!}{\! {}_{\phantom{#3}}\norm{#2}_{#3}\!}}%
}
\newcommand\coloniff{\vcentcolon\!\iff}

\makeatletter
\newcommand*{\transpose}{%
  {\mathpalette\@transpose{}}%
}
\newcommand*{\@transpose}[2]{%
  % #1: math style
  % #2: unused
  \raisebox{\depth}{$\m@th#1\intercal$}%
}
\makeatother

\newcommand{\stcomp}[1]{{#1}^{\mathsf{c}}} % Mengenkomplement
\newcommand{\CC}{\mathbb{C}}
\newcommand{\EE}{\mathbb{E}}
\newcommand{\FF}{\mathbb{F}}
\newcommand{\GG}{\mathbb{G}}
\newcommand{\HH}{\mathbb{H}}
\newcommand{\KK}{\mathbb{K}}
\newcommand{\NN}{\mathbb{N}}
\newcommand{\OO}{\mathbb{O}}
\newcommand{\QQ}{\mathbb{Q}}
\newcommand{\RR}{\mathbb{R}}
\newcommand{\ZZ}{\mathbb{Z}}
\newcommand{\bb}{\mathcal{B}}
\newcommand{\cc}{\mathcal{C}}
\newcommand{\ff}{\mathcal{F}}
\renewcommand{\gg}{\mathcal{G}}
\renewcommand{\ll}{\mathcal{L}}
\newcommand{\nn}{\mathcal{N}}
\newcommand{\vv}{\mathcal{V}}
\newcommand{\zz}{\mathcal{Z}}
\newcommand{\zspace}{V}
\newcommand{\zsigma}{\vv}

\newcommand{\Cb}[1]{C_b(#1)}
\newcommand{\expect}[1]{\EE[#1]}
\newcommand{\condexp}[2]{\EE[#1\,|\,#2]}
\newcommand{\dvar}[1]{\,\mathrm{d}#1}
\newcommand{\dmeas}[2]{\,#1(\mathrm{d}#2)}
% \newcommand{\meas}[2]{\,\mathrm{d}#1(#2)}
\DeclarePairedDelimiter{\sprod}{\langle}{\rangle}	% spitze Klammern
\DeclarePairedDelimiter{\abs}{\lvert}{\rvert}		% Betrag
\DeclareMathOperator{\ggT}{ggT}

\newcommand\phantomarrow[2]{%
  \setbox0=\hbox{$\displaystyle #1\to$}%
  \hbox to \wd0{%
    $#2\mapstochar
     \cleaders\hbox{$\mkern-1mu\relbar\mkern-3mu$}\hfill
     \mkern-7mu\rightarrow$}%
  \,}

\begin{document}

\title{Konvergenzraten des Robbins-Monro-Algorithmus}
\author{Alexander Schlüter}
% \thanks{Dozent: Prof.~Dr.~Dereich, Betreuung: Johannes Blank}
\date{13. Juli 2016}
% Bachelorseminar Markovketten, WS 15/16
% {\let\newpage\relax\maketitle}
\maketitle
\begin{abstract}
  Der Robbins-Monro-Algorithmus ist ein iteratives Verfahren zur Bestimmung von
  Nullstellen einer Funktion, deren Wert nur stochastisch gestört gemessen
  werden kann. Im letzten Vortrag wurde die fast sichere Konvergenz gegen eine
  Nullstelle bewiesen. Ziel dieses Vortrages ist es, eine Aussage über die
  Konvergenzgeschwindigkeit und die Form der Verteilung im Limes ähnlich dem
  zentralen Grenzwertsatz zu beweisen.
\end{abstract}
% \newpage
\tableofcontents
% \newpage

\section{Erinnerungen}
Im Folgenden sei    
\begin{itemize}
\item $(\Omega, \ff, P)$ ein Wahrscheinlichkeitsraum und $(\zspace,\zsigma)$ ein messbarer Raum,
\item $(Z_n)_{n\geq 1}$ eine Folge u.i.v $(\zspace, \zsigma)$-wertiger Zufallsvariablen,
\item $I\subset\RR$ ein abgeschlossenes Intervall oder $I=\RR$, 
\item $X_0$ eine von $(Z_n)_{n\geq 1}$ unabhängige $I$-wertige Zufallsvariable,
\item $\ff_n=\sigma(X_0,Z_1,\dotsc,Z_n)$ und $\FF=(\ff_n)_{n\geq 0}$ eine Filtration,
\item $(\gamma_n)_{n\geq 1}$ eine Folge in $(0,\infty)$ und
\item $H:(I\times \zspace,\bb(I)\otimes\zsigma)\to(\RR,\bb(\RR))$ eine messbare
  Abbildung mit
  \begin{equation*}
    H(x,Z_1)\in\ll^1\quad\text{für alle $x\in I$.}
  \end{equation*}
\end{itemize}

Der \textbf{Robbins-Monro-Algorithmus} ist der $\FF$-adaptierte reelle Prozess
$X=(X_n)_{n\geq 0}$ definiert durch die Rekursion
\begin{equation}
  \label{eq:algo}
  X_{n+1}=X_n+\gamma_{n+1}H(X_n,Z_{n+1})\,,
\end{equation}
wobei wir annehmen, dass die Werte immer in $I$ liegen.

Die \textbf{Erwartungswertfunktion} des Algorithmus ist definiert durch
\begin{equation*}
 h:I\to\RR,\quad h(x)\coloneqq\expect{H(x,Z_1)}
\end{equation*}
und außerdem sei
\begin{equation*}
 g:I\to\overline{\RR}_+,\quad g(x)\coloneqq\expect{H(x,Z_1)^2}.
\end{equation*}

\begin{rem}
  \label{rem:11.2}
  Wir haben gesehen, dass der Robbins-Monro-Algorithmus unter zusätzlichen
  Annahmen fast sicher gegen eine Nullstelle von $h$ konvergiert. Diese Annahmen
  waren
  \begin{enumerate}[label=(\alph*)]
  \item (Downcrossing-Bedingung) $h$ ist stetig und es gibt ein $x_0\in
    h^{-1}(0)$ mit
    \begin{equation*}
      \sup_{x\in I}(x-x_0)h(x)\leq 0\,,
    \end{equation*}
  \item (Beschränkte Störung) $X_0\in\ll^2$ und $g(x)\leq C(1+x^2)$ für eine
    Konstante $C\in\RR_+$ und alle $x\in I$,\label{rem:beschr}
  \item (Abnehmende Schrittweiten) $\sum_{n=1}^\infty \gamma_n=\infty$ und
    $\sum_{n=1}^\infty\gamma_n^2<\infty$.
  \end{enumerate}
\end{rem}
Wir benötigen noch folgende Resultate vom letzten Mal:
\begin{lem}[Doob-Zerlegung]\label{lem:doob}
 Unter den o.g. Bedingungen ist $X$ ein $\ll^2$-Prozess und für die
 Doob-Zerlegung $X=N+A$ in ein Martinal $N=(N_n)_{n\geq 0}$ und einen
 previsiblen Prozess $A=(A_n)_{n\geq 0}$ gilt
 \begin{align*}
  N_n&=X_0+\sum_{j=1}^n\gamma_j(H(X_{j-1},Z_j)-h(X_{j-1})) \\
  \sprod{N}_n&=\sum_{j=1}^n\gamma_j^2(g(X_{j-1})-h(X_{j-1})^2)
 \end{align*}
 für $n\geq 0$. Außerdem ist $N$ ein $\ll^2$-Martingal.
\end{lem}
\begin{lem}[Substitution bei Unabhängigkeit]\label{lem:subst}
 Für $n\geq 1$ und jede messbare Funktion $f:(I\times \zspace,\bb(I)\otimes\zsigma)\to(\RR,\bb(\RR))$ mit
 wohldefiniertem Integral bezüglich $P^{(X_{n-1},Z_n)}$ gilt
 \begin{equation*}
  \condexp{f(X_{n-1}, Z_n)}{\ff_{n-1}}(\cdot)=\int f(X(\cdot), z)\dvar{P^{Z_1}(z)}\quad\text{fast sicher.}
 \end{equation*}
\end{lem}
\begin{proof}
 $Z_n$  ist unabhängig von $\ff_{n-1}$, also ist die Aussage vom letzten Mal anwendbar.
\end{proof}
\section{Vorbereitungen}
\begin{defn}
  % Es sei $K$ ein Markov-Kern und $(K_n)_{n\geq 1}$ eine Folge von Markov-Kernen von
  % $(\Omega, \ff)$ nach $(\RR^d, \bb(\RR^d))$. Außerdem sei $(X_n)_{n\geq 1}$
  % eine Folge von Zufallsvariablen von $(\Omega, \ff)$ nach $(\RR^d, \bb(\RR^d))$.
  % \begin{thmlist}
  % \item Die Folge $(K_n)_{n\geq 1}$ heißt \textbf{schwach konvergent} gegen $K$ und wir
  %   schreiben
  %   \begin{equation*}
  %     K_n\overset{w}{\longrightarrow} K\, ,
  %   \end{equation*}
  %   falls für alle $f\in\Li{P}$ und $h\in\Cb{\RR^d}$ gilt
  %   \begin{equation*}
  %    \lim_{n\to\infty}\int f(\omega)h(x)\, P\otimes K_n(\mathrm{d}(\omega,x)) = \int f(\omega)h(x)\, P\otimes K(\mathrm{d}(\omega,x))\, .
  %   \end{equation*}
  %   \item Die Folge $(X_n)_{n\geq 1}$ heißt \textbf{stabil konvergent} gegen $K$
  %     und wir schreiben
  %     \begin{equation*}
  %       X_n\longrightarrow K
  %     \end{equation*}
  % \end{thmlist}
  Es sei $(Y_n)_{n\geq 1}$ eine Folge von Zufallsvariablen von $(\Omega, \ff)$
  nach $(\RR^d, \bb(\RR^d))$ und $\nu$ ein Wahrscheinlichkeitsmaß auf
  $\bb(\RR^d)$. $(Y_n)_{n\geq 1}$ heißt \textbf{mischend konvergent} gegen $\nu$
  und wir schreiben
  \begin{equation*}
    Y_n\longrightarrow\nu \quad\textit{mischend}, 
  \end{equation*}
  falls für alle $f\in\ll^1(P)$ und $h\in\Cb{\RR^d}$ gilt
  \begin{equation*}
   \lim_{n\to\infty} \expect{f\cdot h(Y_n)} = \int f\dvar{P}\int h\dvar{\nu}\,.
  \end{equation*}
\end{defn}
\begin{rem}
 Durch Wahl von $f=1$ folgt aus der mischenden Konvergenz die bekannte
 Konvergenz in Verteilung. 
\end{rem}
\begin{lem}\label{lem:misch}
  Sei $(Y_n)_{n\geq 1}$ eine Folge von reellen Zufallsvariablen und $\nu$ ein
  Wahrscheinlichkeitsmaß auf $(\RR,\bb(\RR))$ mit $Y_n\to\nu$ \textit{mischend}
  für $n\to\infty$. Außerdem sei $(U_n)_{n\geq 1}$ eine Folge von reellen
  Zufallsvariablen mit $U_n\to 0$ \textit{in Wahrscheinlichkeit}. Dann gilt
\begin{equation*}
 (Y_n,U_n)\to\nu\otimes\delta_0\quad\textit{mischend.}
\end{equation*}
\end{lem}
\begin{proof}
 Siehe \cite[Korollar~5.29(a)]{lpw}.
\end{proof}
\begin{cor}\label{cor:mischsum}
  Sei $(Y_n)_{n\geq 1}$ eine Folge von reellen Zufallsvariablen und $\nu$ ein
  Wahrscheinlichkeitsmaß auf $(\RR,\bb(\RR))$ mit $Y_n\to\nu$ \textit{mischend}
  für $n\to\infty$. Außerdem sei $(U_n)_{n\geq 1}$ eine Folge von reellen
  Zufallsvariablen mit $U_n\to 0$ \textit{in Wahrscheinlichkeit}. Dann gilt
\begin{equation*}
 Y_n + U_n\to\nu\quad\textit{mischend.}
\end{equation*}
\end{cor}
\begin{proof}
  Es seien $f\in\ll^1(P)$ und $h\in\Cb{\RR}$ beliebig. Dann ist
  $h':\RR^2\to\RR$, $h'(y,z)\coloneqq h(y+z)$ in $\Cb{\RR^2}$. Nach
  \cref{lem:misch} gilt
\begin{equation*}
 (Y_n,U_n)\to\nu\otimes\delta_0\quad\textit{mischend.}
\end{equation*}
Nach Definition der mischenden Konvergenz folgt
\begin{align*}
  \expect{f\cdot h(Y_n+U_n)}=\expect{f\cdot h'(Y_n,U_n}&\to\int f\dvar{P}\int h'(y,u) \dvar{\nu\otimes\delta_0(y,u)}\\
&=\int f\dvar{P}\int h(y+0) \dvar{\nu(y)}\,.
\end{align*}
\end{proof}
\begin{thm}[Stabiler CLT]\label{thm:clt}
  Es sei $Y=(Y_n)_{n\geq 0}$ ein $\ll^2$-Martingal bezüglich der Filtration
  $\FF$ und $(\kappa_n)_{n\geq 1}$ eine Folge in $(0,\infty)$ mit $a_n\to 0$. Die
  folgenden Bedingungen seien erfüllt:
 \begin{thmasslist}
 \item Es existiert eine Konstante $v>0$ mit \label{clt:vora}
   \begin{equation*}
     \kappa_n^2\sprod{Y}_n\longrightarrow v\quad\textit{in Wahrscheinlichkeit}
   \end{equation*}
   für $n\to\infty$,
 \item ($\FF$-bedingte Lyapunov-Bedingung) Es existiert ein $\delta >0$, sodass \label{clt:vorb}
   \begin{equation*}
    \kappa_n^{2+\delta} \sum_{j=1}^n\condexp{\abs{\Delta Y_j}^{2+\delta}}{\gg_{j-1}}\to 0\quad\textit{in Wahrscheinlichkeit}
   \end{equation*}
für ${n\to\infty}$.
 \end{thmasslist}
Dann gilt für ${n\to\infty}$:
\begin{equation*}
 \kappa_nY_n\to\nn(0,v)\quad\textit{mischend.} 
\end{equation*}
\end{thm}
\begin{proof}
 Siehe \cite[Satz~5.31 und Bemerkung~5.32(a)]{lpw}.
\end{proof}
\begin{lem}[Diskrete Regel von de~l'Hospital]\label{lem:lhospital}
 Sei $(a_n)_{n\geq 1}$ eine Folge in $\RR$ und $(b_n)_{n\geq 1}$ eine Folge positiver reeller Zahlen, sodass 
 \begin{thmasslist}
 \item 
     $\sum_{n=1}^\infty b_n=\infty$ und
 \item es existiert $c\in\RR$ mit $\lim_{n\to\infty}a_n/b_n=c$
 \end{thmasslist}
 Dann gilt
 \begin{equation*}
  \lim_{n\to\infty} \frac{\sum_{j=1}^na_j}{\sum_{j=1}^nb_j}=c\,.
 \end{equation*}
\end{lem}
\begin{proof}
 Seien $\varepsilon >0$ und $n_0\in\NN$ mit $\abs{a_n/b_n-c}\leq\epsilon$
 für alle $n>n_0$. Dann gilt für $n>n_0$
 \begin{align*}
  \abs*{\frac{\sum_{j=1}^na_j}{\sum_{j=1}^nb_j}-c}&\leq\frac{\sum_{j=1}^nb_j\abs{a_j/b_j-c}}{\sum_{j=1}^nb_j} \\
&=\frac{\sum_{j=1}^{n_0}b_j\abs{a_j/b_j-c}}{\sum_{j=1}^nb_j}+\frac{\sum_{j=n_0+1}^nb_j\abs{a_j/b_j-c}}{\sum_{j=1}^nb_j} \\
&\leq\frac{\sum_{j=1}^{n_0}b_j\abs{a_j/b_j-c}}{\sum_{j=1}^nb_j}+\varepsilon\,.
 \end{align*}
Es folgt
\begin{equation*}
  \limsup_{n\to\infty} \abs*{\frac{\sum_{j=1}^na_j}{\sum_{j=1}^nb_j}-c}\leq\varepsilon
\end{equation*}
und da $\varepsilon$ beliebig war folgt die Behauptung.
\end{proof}
\begin{lem}[Kronecker, WT Lem. 4.28]\label{lem:kronecker}
 Sei $(a_n)_{n\geq 1}$ eine monoton wachsende Folge positiver Zahlen mit
 $\lim_{n\to\infty}a_n=\infty$ und sei $(b_n)_{n\geq 1}$ eine Folge reeller
 Zahlen. Konvergiert $\sum_{j=1}^\infty b_j/a_j$ in $\RR$, so folgt
 \begin{equation*}
  \lim_{n\to\infty} \frac{1}{a_n}\sum_{j=1}^nb_j=0\,.
 \end{equation*}
\end{lem}
\begin{lem}\label{lem:sim}
  \begin{thmlist}
  \item Für reelle Konstanten $b\geq 0$ und $a>-b-1$ existiert $L\in(0,\infty)$,
    sodass\label{lem:sim:prod}
    \begin{equation*}
      \prod_{j=1}^n\left( 1+\frac{a}{b+j} \right) \sim Ln^a\quad\text{für $n\to\infty$.} 
    \end{equation*}
  \item Für $b>-1$ gilt
    \begin{equation*}
      \sum_{j=1}^nj^b \sim \frac{n^{b+1}}{b+1}\quad\text{für $n\to\infty$} 
    \end{equation*}
und für $b=-1$\label{lem:sim:sum}
    \begin{equation*}
      \sum_{j=1}^nj^{-1} \sim \log n\quad\text{für $n\to\infty$.} 
    \end{equation*}
  \end{thmlist}
\end{lem}
\begin{proof}
  \begin{thmlist}
  \item
Die Gammafunktion hat die Eigenschaft 
\begin{equation*}
 \Gamma(t+1)=t\Gamma(t)
\end{equation*}
für eine positive reelle Zahl $t$. Deshalb gilt
\begin{equation*}
  \prod_{j=1}^n\left( 1+\frac{a}{b+j} \right) =\frac{\prod_{j=1}^n(a+b+j)}{\prod_{j=1}^n(b+j)}=
\frac{\Gamma(a+b+n+1)}{\Gamma(a+b+1)}\frac{\Gamma(b+1)}{\Gamma(b+n+1)} 
\end{equation*}
Wir identifizieren den von $n$ unabhängigen Faktor
\begin{equation*}
 L\coloneqq\frac{\Gamma(b+1)}{\Gamma(a+b+1)}
\end{equation*}
und untersuchen den übrigen Term. Stirlings Formel für die Gammafunktion
\begin{equation*}
 \Gamma(t)\sim\sqrt{2\pi} t^{t-1/2}e^{-t},\quad t\to\infty
\end{equation*}
liefert
\begin{align*}
 \frac{\Gamma(a+b+n+1)}{\Gamma(b+n+1)}&\sim (a+b+n+1)^a\left( \frac{a+b+n+1}{b+n+1} \right)^{b+n+1/2}e^{-a} \\
&\sim n^a\left( 1+\frac{a}{b+n+1} \right)^{b+n+1}e^{-a}\,. %\sim n^a
\end{align*}
Im letzten Schritt wurde genutzt, dass
\begin{equation*}
 \lim_{n\to\infty}\left( \frac{a+b+n+1}{n} \right)^a=1\quad\text{und}\quad\lim_{n\to\infty}\sqrt{\frac{b+n+1}{a+b+n+1}}=1\,.
\end{equation*}
Schließlich folgt mit
\begin{equation*}
 \lim_{n\to\infty}\left( 1+\frac{a}{b+n+1} \right)^{b+n+1}=e^a
\end{equation*}
die Behauptung.
\item Falls $-1<b\leq 0$, so ist die Abbildung $(0,\infty)\to\RR$, $x\mapsto
  x^b$ monoton fallend. Also gilt für $j\in\NN$
  \begin{equation*}
   (j+1)\leq\int_j^{j+1}x^b\dvar{x}\leq j^b
  \end{equation*}
  und deshalb
  \begin{equation*}
   \frac{(n+1)^{b+1}-1}{b+1}=\int_1^{n+1}x^b\dvar{x}\leq\sum_{j=1}^nj^b\leq\int_0^nx^b\dvar{x}=\frac{n^{b+1}}{b+1}\,.
  \end{equation*}
Die Behauptung folgt, da
\begin{equation*}
 \lim_{n\to\infty} \frac{(n+1)^{b+1}-1}{b+1}\cdot\frac{b+1}{n^{b+1}}=1\,.
\end{equation*}
Mit demselben Argument bekommt man den Fall $b=-1$:
\begin{equation*}
   \log(n+1)=\int_1^{n+1}x^b\dvar{x}\leq\sum_{j=1}^nj^b\leq 1+\int_1^nx^b\dvar{x}=1+\log(n)
\end{equation*}
und
\begin{equation*}
 \lim_{n\to\infty}\frac{\log(n+1)}{\log(n)} = 1 = \lim_{n\to\infty}\frac{1+\log(n)}{\log(n)}\,.
\end{equation*}
  Falls $b>0$, so ist $x\mapsto x^b$ monoton wachsend, also
  \begin{equation*}
   j^b\leq\int_j^{j+1}x^b\dvar{x}\leq (j+1)^b
  \end{equation*}
  und die Abschätzung ist analog zu oben.
  \end{thmlist}
\end{proof}
\section{Konvergenzraten}
\begin{thm}[Konvergenzraten, stabiler CLT]
 \label{thm:haupt}
 Seien $\gamma_n=C_1/(C_2+n)$ für $n\geq 1$ mit reellen Konstanten $C_1>0$,
 $C_2\geq 0$ und $x_0\in h^{-1}(0)$. Die folgenden Bedingungen seien erfüllt:
 \begin{thmasslist}
 \item $X_n\to x_0$ f.s. für $n\to\infty$.
 \item $X_0\in\ll^2$ und $g(x)\leq C(1+x^2)$ für eine Konstante $C\in\RR_+$ und
   alle $x\in I$,\label{thm:haupt:b}
 \item $g$ sei stetig in $x_0$, h sei differenzierbar in $x_0$, $h'(x_0)<0$ und\label{thm:haupt:c}
   \begin{equation*}
    h(x) = h'(x_0)(x-x_0) + \mathcal{O}((x-x_0)^2)\quad\text{für $x\to x_0$}\,,
   \end{equation*}
 \item Es existieren $\varepsilon,\delta >0$ sodass $H(x,Z_1)\in\ll^{2+\delta}$ für
   alle $x\in I$ und\label{thm:haupt:d} 
   \begin{equation*}
    \sup_{\abs{x-x_0}\leq\varepsilon}\expect{\abs{H(x,Z_1)}^{2+\delta}}<\infty\,.
   \end{equation*}
 \end{thmasslist}
 Dann gilt für $n\to\infty$:
 \begin{enumerate}[label=(\roman*)]
 \item Falls $\abs{h'(x_0)}C_1>1/2$:\label{thm:haupt:i}
   \begin{equation*}
    \sqrt{n}(X_n-x_0)\to\nn\left( 0,\frac{g(x_0)C_1^2}{2\abs{h'(x_0)}C_1-1} \right)\quad\textit{mischend,}
   \end{equation*}
 \item Falls $\abs{h'(x_0)}C_1=1/2$:
   \begin{equation*}
    \sqrt{\frac{n}{\log n}}(X_n-x_0)\to\nn\left( 0,g(x_0)C_1^2 \right)\quad\textit{mischend,}
   \end{equation*}
 \item Falls $\abs{h'(x_0)}C_1<1/2$: 
   \begin{equation*}
     n^{\abs{h'(x_o)}C_1}(X_n-x_0)\to\xi \quad\text{f.s.}
   \end{equation*}
   für eine reelle Zufallsvariable $\xi$, die von $X_0$ abhängt.
 \end{enumerate}
\end{thm}
\begin{rem}
  Downcrossing etc.
\end{rem}
\begin{proof}
  Hier wird nur ein Teil vom Fall $\abs{h'(x_0)}C_1>1/2$ bewiesen. Für
  den vollen Beweis siehe \cite[Satz~11.4]{lpw}.
  
Wie wir letzte Woche gesehen haben, folgt aus \cref{thm:haupt:b}, dass $X$ ein
$\ll^2$-Prozess ist. Sei
\begin{equation*}
 a\coloneqq -h'(x_0)=\abs{h'(x_o)}\,. 
\end{equation*}
Nach \ref{thm:haupt:c} gilt $a>0$. Man wähle $n_0\in\NN$ mit $a\gamma_{n_0}<1$ und
definiere für $n\geq 0$
\begin{equation*}
 \beta_n\coloneqq\prod_{j=n_0}^n(1-a\gamma_j) =\prod_{j=1}^{n-n_0+1}(1-a\gamma_{j+n_0-1})\,.
\end{equation*}
Insbesondere gilt also $\beta_0=\dotsb=\beta_{n_0-1}=1$ und nach \cref{lem:sim:prod}
\begin{equation}\label{eq:beta}
 \beta_n\sim Ln^{-aC_1}\,
\end{equation}
für $n\to\infty$ mit einer Konstanten $L\in(0,\infty)$. Wegen $\gamma_n\sim
C_1/n$ folgt
\begin{equation}
  \frac{\gamma_n}{\beta_n}\sim\frac{C_1}{L}n^{aC_1-1}\label{eq:gammabeta}
\end{equation}
Mit der Doob-Zerlegung $X=N+A$ aus \cref{lem:doob} bekommen wir
\begin{align*}
  \sqrt{n}(X_n-x_0)&=\sqrt{n}\beta_n\left(X_0-x_0+\sum_{j=1}^n\frac{1}{\beta_j}(X_j - X_{j-1} + X_{j-1})-\frac{1}{\beta_{j-1}}X_{j-1}-\frac{x_0}{\beta_j}+\frac{x_0}{\beta_{j-1}}\right)\nonumber \\
&=\phantom{+}\sqrt{n}\beta_n(X_0-x_0)\label{eq:1}\\
&\phantom{=}+\sqrt{n}\beta_n\sum_{j=1}^n\frac{1}{\beta_j}\Delta N_j\\
&\phantom{=}+\sqrt{n}\beta_n\sum_{j=1}^n\left(\frac{1}{\beta_j}\Delta A_j+(X_{j-1}-x_0)(\frac{1}{\beta_j}-\frac{1}{\beta_{j-1}})\right)\,. 
\end{align*}
Wir wollen die drei Summanden einzeln auf ihr Verhalten für $n\to\infty$
untersuchen. Definiere dazu für $n\geq 0$
\begin{align*}
 M_n&\coloneqq \sum_{j=1}^n\frac{1}{\beta_j}\Delta N_j\\
B_n&\coloneqq\sum_{j=1}^n\left(\frac{1}{\beta_j}\Delta A_j+(X_{j-1}-x_0)(\frac{1}{\beta_j}-\frac{1}{\beta_{j-1}})\right)\,.
\end{align*}
Da $\sqrt{n}\beta_n\sim Ln^{-aC_1+1/2}\to 0$ gilt auch
$\sqrt{n}\beta_n(X_0-x_0)\to 0$ punktweise.

Es gilt auch
$\sqrt{n}\beta_nB_n\to 0$ fast sicher, allerdings ist der Beweis nicht trivial
und kann nachgelesen werden in \cite[Satz~11.4]{lpw}. Er basiert auf der Idee,
die Zuwüchse $\Delta B_n$ für große $n$ gegen $(X_n-x_0)^2$ abzuschätzen. Dazu
wird \cref{thm:haupt:c} genutzt.

Im Folgenden wird mit dem stabile CLT~\ref{thm:clt} gezeigt, dass
\begin{equation}\label{eq:mterm}
 \sqrt{n}\beta_nM_n\to\nn\left( 0,\frac{g(x_0)C_1^2}{2aC_1-1} \right)\quad\textit{mischend.}
\end{equation}
% denn dann folgt die Behauptung mittels \cref{cor:mischsum}. \\
Wähle also
$\kappa_n=\sqrt{n}\beta_n$, dann gilt nach \cref{eq:beta} und wegen $aC_1>1/2$
\begin{equation*}
 \kappa_n\sim Ln^{-aC_1+1/2}\to 0\quad\text{für $n\to\infty$.} 
\end{equation*}
$M=(M_n)_{n\geq 0}$ ist ein $\ll^2$-Martingal, da $N$ eins ist.

Wir wollen nun \cref{clt:vora} des CLT überprüfen. Es gilt nach der
Darstellung für $\sprod{N}$ aus \cref{lem:doob}
\begin{align*}
 \Delta\sprod{M}_n&=\condexp{(\Delta M_n)^2}{\ff_{n-1}}=\frac{1}{\beta_j^2}\condexp{(\Delta N_n)^2}{\ff_{n-1}}\\
&=\frac{\gamma_j^2}{\beta_j^2}(g(X_{j-1})-h(X_{j-1})^2)
\end{align*}
für $n\geq 0$. Wegen der Stetigkeit von $h$ und $g$ in $x_0$ gilt
\begin{equation*}
 g(X_{n-1})-h(X_{n-1})^2\to g(x_0)-h(x_0)^2=g(x_0)\quad\text{f.s.,}
\end{equation*}
und zusammen mit \cref{eq:gammabeta} ergibt sich
\begin{equation*}
 \frac{\Delta\sprod{M}_n}{n^{2aC_1-2}}\to\frac{g(x_0)C_1^2}{L^2}\quad\text{f.s.} 
\end{equation*}
für $n\to\infty$.
Da wir $aC_1>1/2$ annehmen, ist $2aC_1-2>-1$ und deshalb
\begin{equation*}
 \sum_{j=1}^\infty j^{2aC_1-2}=\infty\,. 
\end{equation*}
Somit können wir die diskrete Regel von l'Hospital~\ref{lem:lhospital} anwenden
und bekommen
\begin{equation*}
 \frac{\sprod{M}_n}{\sum_{j=1}^n j^{2aC_1-2}}\to\frac{g(x_0)C_1^2}{L^2}\quad\text{f.s.}
\end{equation*}
für $n\to\infty$. Aus \cref{lem:sim:sum} und \cref{eq:beta} kennen wir die
asymptotischen Äquivalenzen
\begin{equation*}
  \frac{1}{\sum_{j=1}^n j^{2aC_1-2}}\overset{\ref{lem:sim}}{\sim}\frac{2aC_1-1}{n^{2aC_1-1}}\overset{\ref{eq:beta}}{\sim}\frac{2aC_1-1}{L^2}n\beta_n^2\
\end{equation*}
d.h. insgesamt
\begin{equation*}
 \kappa_n^2\sprod{M}_n\to\frac{g(x_0)C_1^2}{2aC_1-1}\quad\text{f.s.}
\end{equation*}

Zur Anwendung des CLT fehlt uns noch die bedingte Lyapunov-Bedingung. Es gilt
für $j\geq 1$
\begin{equation*}
 \Delta M_j=\frac{\gamma_j}{\beta_j}(H(X_{j-1},Z_j)-h(X_{j-1}))\,,
\end{equation*}
also mit dem $\delta$ aus \cref{thm:haupt:d}
\begin{equation}\label{eq:lyasup}
 \sum_{j=1}^n\condexp{\abs{\Delta M_j}^{2+\delta}}{\ff_{j-1}}\leq\sum_{j=1}^n\left( \frac{\gamma_j}{\beta_j} \right)^{2+\delta}\sup_{k\geq 1}\condexp{\abs{H(X_{k-1}, Z_k)-h(X_{k-1})}^{2+\delta}}{\ff_{k-1}}\,.
\end{equation}
Das Supremum über die bedingten Erwartungswerte kann gebildet werden, weil die
Indexmenge abzählbar ist. Für $n\to\infty$ gilt wegen $\gamma_n\sim C_1/n$
\begin{equation*}
 (\sqrt{n}\beta_n)^{2+\delta} \left( \frac{\gamma_n}{\beta_n} \right)^{2+\delta}=n^{1+\delta/2}\gamma_n^{2+\delta}\sim C_1^{2+\delta}n^{-1-\delta/2}
\end{equation*}
und deshalb
\begin{equation*}
 \sum_{n=1}^\infty \kappa_n^{2+\delta}\left( \frac{\gamma_n}{\beta_n} \right)^{2+\delta}<\infty\,.
\end{equation*}
Mit dem Kronecker-Lemma~\ref{lem:kronecker} folgt
\begin{equation*}
 \kappa_n^{2+\delta} \sum_{j=1}^n\left( \frac{\gamma_n}{\beta_n} \right)^{2+\delta}\to 0\,.
\end{equation*}
Es soll nun das Supremum in \cref{eq:lyasup} mithilfe von \cref{thm:haupt:d} abgeschätzt werden.
Mittels Substitution bei Unabhängigkeit \ref{lem:subst} gilt für $n\geq 1$
\begin{align*}
  \condexp{\abs{H(X_{n-1},Z_n)-h(X_{n-1})}^{2+\delta}}{\ff_{n-1}}&=\int\abs{H(X_{n-1},z)-h(X_{n-1})}^{2+\delta}\dvar{P^{Z_1}(z)} \\
&=\norm{H(X_{n-1},\cdot)-h(X_{n-1})}_{2+\delta}^{2+\delta}
\end{align*}
 mit der Norm auf $\ll^{2+\delta}(P^{Z_1})$. Durch die Dreiecksungleichung kann
 man abschätzen
 \begin{equation*}
   
 \end{equation*}
\end{proof}
\begin{rem}
 Optimale Schrittweite 
\end{rem}
\begin{exmp}
 Mittelwert und Quantil 
\end{exmp}
\printbibliography

\end{document}